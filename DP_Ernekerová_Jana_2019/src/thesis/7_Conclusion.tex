\chapter{Conclusion}
\label{chap:conclusion}

This thesis aimed to study the KRACK attacks and analyze its traffic. It also proposes a tool for testing device vulnerability and creates a system for KRACK detection in real-time. 

The KRACK attacks exploit a vulnerability found in the standard 802.11i defining WPA2 protocol. It is the only Wi-Fi data security protocol that was considered secure for more than a decade. Most of us use Wi-Fi every day and often do not think about data we send over it. Thus, when Mathy Vanhoef found this vulnerability and published it in October 2017, it became a media stunt. The vast majority of Wi-Fi devices were affected by this vulnerability. Also, it meant that all data we send over Wi-Fi might be possibly decrypted. 

In this work, we studied the principle of the KRACK attacks vulnerability. For this purpose, it was necessary to become acquainted with parts of the Wi-Fi standard and with research behind it. Based on what we studied, the thesis describes the process of Wi-Fi client connection, and the principle of the KRACK attacks vulnerability, its practical impact, and countermeasures. We found that the implementation of the attack is quite complicated, but we modified the proof-of-concept scripts and managed to perform the attack against a few vulnerable mobile devices. Thus, we were able to monitor both standard, and malicious traffic of the 4-way handshake and data frames sent after it. We have created a lab consisting of five Wi-Fi NICs that were switched into monitor mode, and tools to capture and transmit Wi-Fi frames. The lab was used for practical experiments described in this thesis. Based on what we studied and analyzed, we made a list of characteristics that make the attack detectable. The results are also contained in this thesis.

We have proposed the tool for testing device vulnerability and used an existing solution for testing more than twenty mobile devices. We found some of them vulnerable and also, we found a device that still reinstalls the all-zero-key. Besides, we proposed a system for detection of the KRACK attack in real-time. The developed scripts detect the attack against the 4-way handshake by detecting retransmission of the third message of the 4-way handshake and following reinstallation of the session key. In a typical environment with not extremely high interference, the script works reliably and does not trigger any false alarms.

Even though the thesis has met the set goals, the extension of the detection system might be a beneficial future work. Also, the analysis of other handshakes from the standard for vulnerabilities could help improve Wi-Fi security. The topic is extensive, and I found this work a decent basis for future extension.