\chapter{Introduction}
% Aktuálnost tématu („Používání mobilních zařízení se stalo běžnou součástí našeho života ...“)

Many people these days cannot imagine their lives without the Internet. Not~only that the communication and sharing information with people around the world is the simplest it has ever been, but it also became our helper in everyday situations. One of the reasons why this is possible is a speedy development of the Wi-Fi technology. In nearly twenty years of its development, it has become a standard that almost every mobile device can connect to the Wi-Fi. Besides, its installation is cheaper and easier than the installation of the wired networks and provides its users higher comfort when using it. Also, it is cheaper than cellular data which can be a good alternative. There are not only mobile devices that can connect to these wireless networks today, but also televisions, security cameras, and other home appliances. We got used to having free Wi-Fi available everywhere and to everyone. Unfortunately, this technology also brings security risks, because anyone in the signal range can potentially monitor our communication. 

To keep data transmitted through Wi-Fi private, we use data encryption. In the original standard, there was an optional security algorithm for data encryption defined; it was called \gls{wep}. Unfortunately, it was broken quite shortly after releasing the initial document, already in 2001 \cite{Cam-Winget03, finalNailWEP, MEKHAZNIA_2015}. The \gls{wpa} protocol temporarily replaced it. Although it is still used by some devices mostly because of the backward hardware compatibility, it is not considered to be secure~\cite{finalNailWEP}. In~2004~\cite{ieee802.11i_2004}, it was finally replaced by the \gls{wpa2} protocol. For more than a decade, this protocol was considered secure, assuming usage of a strong password.

The Wi-Fi security is becoming a more and more important topic. Exposure of the personal data could make our life deeply uncomfortable. But for companies, the breach of their network security can have an even more profound impact. It could lead to an exposure of their know-how, their corporate strategy, or the destruction of the company reputation.

% Význam tématu pro společnost (komunitu) („Výsledek práce bude prospěšný pro hendikepované uživatele...“)

\textit{The KRACK attacks} is an abbreviation standing for the Key Reinstallation Attacks. This family of attacks exploits the vulnerability found in the 802.11i amendment which defines the protocols whose implementation is  certified under the trademark WPA and WPA2. Specifically, it targets mainly the initial 4-way handshake. This handshake is used for mutual authentication and negotiation of the session key between a client and an access point. Both, the WPA2 and WPA protocols use it. All of the modern protected \mbox{Wi-Fi} devices use some version of these protocols for data encryption. Therefore, most of these devices are affected by this vulnerability. These attacks trick the victim to reinstall an already-in-use key and so, let the adversary replay, decrypt and possibly forge frames. After the publication of the attacks, in October 2017 \cite{VA17}, people were assured that all of the vendors were notified about the vulnerability in advance and the patches they are about to release are going to protect the devices. Unfortunately, there are many devices in the market that are not supported or cannot be patched.

% Motivace volby tématu („Téma jsem si zvolil, neboť problém... nebyl uspokojivě vyřešen...“)

The detection of the KRACK attacks in real-time can help to identify affected devices in a Wi-Fi network. Then, these devices could be updated (if possible), replaced, or in case of a network with a lot of passing clients, the suspected ones could be possibly de-authenticated from the network to avoid the attacks that target the access point. The detection can also help to re-examine the question that has not yet been answered, whether these attacks are being carried out in the real world or not. And if they are, then, on what scale. It could be an excellent future study or experiment which would lead to the creation of a statistic how often are these attacks exploited. Results could be an illustrative instrument to convince common users to be more aware of network security. 

% Zaměření práce („V práci se zabývám analýzou, návrhem a implementací aplikace ..., použil jsem metodu, postup ....“)
%Vytvořte cíl vaší bakalářské práce:
%Rešeršní části práce: „Cílem rešeršní části práce je ...“ „Získání přehledu současné literatury o ...“
%„Seznámení se základními principy ...“
%„Vysvětlení principů z oblasti (tématiky) ...“
%„Analýza problematiky ...“
%„Studium základních pojmů z oboru...“ „Analýza současných řešení ...“
%Praktické části práce: „Cílem praktické části práce je ...“ „Návrh a implementace aplikace, rozhraní ...“
%„Vytvoření studie ...“
%„Navržení efektivního algoritmu ...“
%„Prozkoumání stávajících metod ... a následný návrh a implementace... “

This thesis analyzes the principle of the KRACK vulnerability, the traffic generated during the attack against the 4-way handshake and necessary countermeasures for avoiding the attack in general. In this manner, it follows up research about the KRACK attacks by Mathy Vanhoef and Frank Piessens from the Catholic University of Leuven \cite{VA_ccs2017, VA_ccs2018, VA17, VA18}. Besides, the thesis maps currently available tools for detection of a device vulnerability to this attack. We tested a couple of devices by the publicly available device vulnerability detection tool and notified their users about the results. The primary purpose of the thesis is to analyze the principle and the traffic generated during the attack and propose a way how to detect it. For this purpose, the standard Wi-Fi traffic was studied first along with related parts of the Wi-Fi standard. It was necessary for understanding the attack because the vulnerability is in the standard itself. We briefly describe all the vulnerable handshakes, but we chose to focus on the attack on the 4-way handshake. We decided so because this attack has the most significant impact and also because the implementation of the attack was necessary for the detection. Based on the analytical part of the work, a system for detection of the attack to the 4-way handshake in real-time is designed and implemented. The implemented system runs on an independent device physically located in a Wi-Fi network. The system detects triggered retransmissions of the specific frames on the network running in 2.4\,GHz band and then logs these security incidents. The system is then verified by performing the attack on the tested network, and the results are expressed as a ratio of false positives and correctly detected events.

% Představení struktury práce – návaznost jednotlivých kapitol: (Nejdříve se v části 1 věnuji ..., z čehož vyplyne potřeba ujasnit si otázku ..., které se věnuji v části 2.)
% Návaznost na jiné bakalářské práce, projekty („Tato práce navazuje na bakalářskou práci ... studenta ..., která se zabývá ...“) 

The structure of this master’s thesis is as follows: In Chapter~\ref{chap:wifi}, we describe relevant parts of the 802.11 standard --- structure of work-related frames and the handshakes that are vulnerable to the attacks. In Chapter~\ref{chap:krack}, we describe the principle of the KRACK attack vulnerability. Also, we describe its practical impact and countermeasures that should be taken to avoid its exploit. In Chapter~\ref{chap:trafficMonitoring}, we describe the traffic generated during the standard 4-way handshake and during the KRACK attack against it. Based on the traffic and analysis from \ref{chap:krack} and \ref{chap:trafficMonitoring}, we create a list of characteristics we can use for detection of the KRACK attack. In Chapter~\ref{chap:design}, we first propose a software testing tool that can check whether a Wi-Fi device under test is resistant against this type of attack, second we design a system for detection of KRACK attacks in real-time. Because the topic is extensive, we focused on the attack against the 4-way handshake. In Chapter~\ref{chap:implementation}, we describe our implementation of such a system and in Chapter~\ref{chap:testing}, we test the system and evaluated results. Besides, in \ref{chap:testing}, we test several devices on their vulnerability.