\chapter{Assigned CVE identifiers}
\label{app:cve}
The following Common Vulnerabilities and Exposures (CVE) identifiers were assigned to track which products are affected by specific instantiations of the key reinstallation attack:

\begin{itemize}
\item CVE-2017-13077: Reinstallation of the pairwise encryption key (PTK-TK) in the 4-way handshake.
\item CVE-2017-13078: Reinstallation of the group key (GTK) in the 4-way handshake.
\item CVE-2017-13079: Reinstallation of the integrity group key (IGTK) in the 4-way handshake.
\item CVE-2017-13080: Reinstallation of the group key (GTK) in the group key handshake.
\item CVE-2017-13081: Reinstallation of the integrity group key (IGTK) in the group key handshake.
\item CVE-2017-13082: Accepting a retransmitted Fast BSS Transition (FT) Reassociation Request and reinstalling the pairwise encryption key (PTK-TK) while processing it.
\item CVE-2017-13084: Reinstallation of the STK key in the PeerKey handshake.
\item CVE-2017-13086: reinstallation of the Tunneled Direct-Link Setup (TDLS) PeerKey (TPK) key in the TDLS handshake.
\item CVE-2017-13087: reinstallation of the group key (GTK) when processing a Wireless Network Management (WNM) Sleep Mode Response frame.
\item CVE-2017-13088: reinstallation of the integrity group key (IGTK) when processing a Wireless Network Management (WNM) Sleep Mode Response frame.
Note that each CVE identifier represents a specific instantiation of a key reinstallation attack. This means each CVE ID describes a specific protocol vulnerability, and therefore many vendors are affected by each individual CVE ID.
\end{itemize}